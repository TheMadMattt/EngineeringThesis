\chapter{Testy aplikacji}
\section{Testy systemowe}
Testy systemowe mają na celu przetestowanie systemu jako całości. Testowanie aplikacji odbywa się poprzez realizację konkretnych scenariuszy. Testy systemowe przeprowadzane są w celu zweryfikowania aplikacji pod kątem zgodności z wymaganiami funkcjonalnymi oraz niefunkcjonalnymi. Poniżej zostało przedstawionych kilka scenariuszy testowych:
\\
\begin{enumerate}
    \item Scenariusz dodawania nowej faktury: \\
        \textbf{Kroki:}
        \begin{enumerate}
            \item Uruchomienie aplikacji.
            \item Kliknięcie przycisku Dodaj fakturę (przycisk z plusem na pasku narzędzi)
            \item Wypełnienie wszystkich wymaganych pól. 
            \item Kliknięcie przycisku z plusem w celu otworzenia menu wyboru.
            \item Wybranie ikonki z dyskietką, w celu zapisania faktury.
            \item Kliknięcie prawym przyciskiem myszy na nowo dodanej fakturze.
            \item Wybranie opcji ,,Edytuj fakturę''.
            \item Sprawdzenie czy wszystkie dane wprowadzone przez użytkownika zgadzają się z wyświetlonymi.
        \end{enumerate}
        \textbf{Oczekiwany rezultat:}
        \begin{itemize}
            \item Dane wpisane przez użytkownika zgadzają się z danymi wyświetlonymi na ekranie.
            \item Faktura została poprawnie dodana do bazy danych.\\
        \end{itemize}
    \item Scenariusz dodawania nowego kontrahenta: \\
        \textbf{Kroki:}
        \begin{enumerate}
            \item Uruchomienie aplikacji.
            \item Kliknięcie w zakładkę kontrahenci.
            \item Kliknięcie przycisku Dodaj kontrahenta (przycisk z plusem na pasku narzędzi).
            \item Wybranie rodzaju klienta: Kontrahent.
            \item Wypełnienie wszystkich wymaganych pól.
            \item Kliknięcie przycisku ,,Dodaj +''.
            \item Kliknięcie prawym przyciskiem myszy na nowo dodanym kontrahencie.
            \item Wybranie opcji ,,Edytuj kontrahenta''.
            \item Sprawdzenie czy wszystkie dane wprowadzone przez użytkownika zgadzają się z wyświetlonymi.
        \end{enumerate}
        \textbf{Oczekiwany rezultat:}
        \begin{itemize}
            \item Dane wpisane przez użytkownika zgadzają się z danymi wyświetlonymi na ekranie.
            \item Kontrahent została poprawnie dodana do bazy danych.
        \end{itemize}
    \item Scenariusz usuwania faktury: \\
        \textbf{Warunki początkowe: } \\
        Dodana minimum jedna faktura na liście.\\
        \textbf{Kroki: }
        \begin{enumerate}
            \item Wybranie konkretnej faktury z listy.
            \item Kliknięcie prawym przyciskiem.
            \item Wybranie opcji ,,Usuń fakturę''.
            \item Potwierdzenie usunięcia faktury kliknięciem w przycisk ,,Tak''
        \end{enumerate}
        \textbf{Oczekiwany rezultat:}
        \begin{itemize}
            \item Faktura została usunięta z listy.
            \item Faktura została usunięta z bazy danych.
            \item Pozycje powiązane z fakturą zostały usunięte z bazy danych.
        \end{itemize}
\end{enumerate}

Wszystkie powyższe testy zostały przeprowadzone według ukazanych kroków. Testy dały rezultaty zgodne z oczekiwanymi. Nie został napotkany żaden problem.

\section{Testy akceptacyjne}
Aplikacja została wysłana do kilku osób nie związanych z projektem. Osoby te dostały zadania, które mają wykonać w aplikacji:
\begin{itemize}
    \item Dodanie faktury
    \item Dodanie kontrahenta
    \item Dodanie sprzedawcy
    \item Edycja faktury
    \item Dodanie do faktury kilku pozycji
    \item Wygenerowanie pliku PDF istniejącej faktury
\end{itemize}

Podczas wykonywania zadań zostało wykrytych kilka błędów:
\begin{itemize}
    \item Jeżeli w aplikacji nie było kontrahentów/sprzedawców aplikacja się zamykała
    \item Podczas dodawania pozycji do faktury, mając angielski system, waluty były źle formatowane
    \item Aplikacja źle wyliczała kolejny numer porządkowy faktury.
\end{itemize}

Wszystkie błędy zostały bezproblemowo i sprawnie naprawione. Po poprawieniu błędów, osobom testującym aplikację, udało się bez problemu wykonać wszystkie powyższe zadania.