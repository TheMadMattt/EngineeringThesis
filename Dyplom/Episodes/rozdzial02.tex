\chapter{Analiza wymagań użytkownika}

Aplikacja skierowana jest do każdego użytkownika lub firmy, który potrzebuje systemy, w którym będzie mógł wprowadzać faktury VAT lub pro forma. Ponadto oprogramowanie umożliwa także generowanie PDFów z wybranych przez użytkownika faktur. Interfejs aplikacji jest bardzo intuicyjny oraz prosty w obsłudze, także każda osoba, będzie potrafiła z niego korzystać.

\section{Wymagania funkcjonalne}
\begin{enumerate}
    \item Użytkownik ma możliwość dodania kontrahenta/sprzedawcę \\
    \textbf{Opis: } Każdy użytkownik po wejściu w odpowiednią zakładkę ma możliwość dodania nowego klienta. Klient może zostać dodany również z widoku dodawania faktury.\\
    \textbf{Przebieg: }
    \begin{enumerate}
        \item Użytkownik uzupełnia wymagane pola (nazwa firmy, miasto, ulica, numer domu) oraz może uzupełnić pola opcjonalne (kod pocztowy, numer lokalu, numer telefonu, NIP, REGON, numer konta bankowego, uwagi).
        \item Aplikacja sprawdza poprawność wpisanych danych (w przypadku błędnie podanych, wyświetla stosowny komunikat).
        \item Kontrahent/Sprzedawca zostaje dodany do bazy danych oraz wyświetla się na liście dostępnych już klientów.\\
    \end{enumerate}
    
    \item Użytkownik ma możliwość usunięcia kontrahenta/sprzedawcy\\
    \textbf{Opis: } Każdy użytkownik ma możliwość wybrania dowolnego sprzedawcy/kontrahenta i usunięcia go.\\
    \textbf{Przebieg: } 
    \begin{enumerate}
        \item Użytkownik wybiera sprzedawcę/kontrahenta z listy dostępnych.
        \item Użtykownik klika w odpowiedni przycisk i potwierdza usunięcie.\\
    \end{enumerate}
    
    \item Użytkownik ma możliwość edytowania kontrahenta/sprzedawcy\\
    \textbf{Opis: } Każdy użytkownik po wybraniu konktretnego kontrahenta/sprzedawcy ma możliwość edytowania go.\\
    \textbf{Przebieg: }
    \begin{enumerate}
        \item Po wybraniu konkretnego sprzedawcy/kontrahenta użytkownikowi ukazuje się menu podglądu.
        \item Po przejściu do trybu edycji użytkownik może edytować wszystkie pola.
        \item Użytkownik zatwierdza wprowadzone zmiany.
        \item Wprowadzone zmiany zostają zaktualizowane na liście.\\
    \end{enumerate}
    
    \item Użytkownik ma możliwość dodania faktury \\
    \textbf{Opis: } Każdy użytkownik po wejściu w odpowiednią zakładkę ma możliwość dodania nowej faktury.\\
    \textbf{Przebieg: }
    \begin{enumerate}
        \item Użytkownik uzupełnia wymagane pola (kontrahent, sprzedawca, data wystawienia faktury, termin płatności) oraz może uzupełnić pola opcjonalne (kiedy faktura została opłacona, dodanie nowych pozycji do faktury, uwagi).
        \item Numer faktury jest kolejnym numerem porządkowym ostatniej faktury (np. ostatnia faktura 1/2019, następna faktura 2/2019), przy rozpoczęciu nowego roku, numeracja faktur rozpoczyna się od początku.
        \item Aplikacja sprawdza poprawność wpisanych danych (w przypadku błędnie podanych, wyświetla stosowny komunikat).
        \item Faktura zostaje dodana do bazy danych oraz wyświetla się na liście dostępnych już faktur.\\
    \end{enumerate}
    
    \item Użytkownik ma możliwość usunięcia faktury\\
    \textbf{Opis: } Każdy użytkownik ma możliwość wybrania dowolnej faktury i usunięcia jej.\\
    \textbf{Przebieg: } 
    \begin{enumerate}
        \item Użytkownik wybiera fakturę z listy dostępnych.
        \item Użtykownik klika w odpowiedni przycisk i potwierdza usunięcie.
        \item W przypadku, gdy użytkownik usuwa ostatnią fakturę z danego roku, numeracja faktur sie cofa (np. mamy faktury 1/2019, 2/2019, 3/2019, użytkownik usuwa fakturę 3/2019, czyli następna nowa faktura będzie miała numer usuniętej faktury 3/2019).\\
    \end{enumerate}
    
    \item Użytkownik ma możliwość edytowania faktury\\
    \textbf{Opis: } Każdy użytkownik po wybraniu konktretnej faktury ma możliwość edytowania jej.\\
    \textbf{Przebieg: }
    \begin{enumerate}
        \item Po wybraniu konkretnej faktury, użytkownikowi ukazuje się menu podglądu.
        \item Po przejściu do trybu edycji użytkownik może edytować wszystkie pola.
        \item Użytkownik zatwierdza wprowadzone zmiany.
        \item Wprowadzone zmiany zostają zaktualizowane na liście.\\
    \end{enumerate}
    
        \item Użytkownik ma możliwość dodania pozycji na fakturze \\
    \textbf{Opis: } Każdy użytkownik po utworzeniu nowej faktury, ma możliwość dodania nowej pozycji do faktury.\\
    \textbf{Przebieg: }
    \begin{enumerate}
        \item Użytkownik uzupełnia wymagane pola (nazwa produktu, ilość, VAT [\%], jednostka, cena netto lub brutto) i opcjonalne pola PKWiU oraz uwagi.
        \item Aplikacja sprawdza poprawność wpisanych danych (w przypadku błędnie podanych, wyświetla stosowny komunikat).
        \item Pozycja zostaje dodana do faktury oraz wyświetla się na liście pozycji.
    \end{enumerate}
    
    \item Użytkownik ma możliwość usunięcia pozycji z faktury\\
    \textbf{Opis: } Każdy użytkownik ma możliwość wybrania dowolnej pozycji znajdującej się na fakturze i usunięcia jej.\\
    \textbf{Przebieg: } 
    \begin{enumerate}
        \item Użytkownik wybiera pozycję z otwartego okna tworzenia/edycji faktury. 
        \item Użtykownik klika w odpowiedni przycisk i potwierdza usunięcie. \\
    \end{enumerate}
    
    \item Użytkownik ma możliwość edytowania pozycji z faktury\\
    \textbf{Opis: } Każdy użytkownik po wybraniu konktretnej pozycji znajdującej się na fakturze ma możliwość edytowania jej.\\
    \textbf{Przebieg: }
    \begin{enumerate}
        \item Użytkownik wybiera konkretna pozycję z faktury.
        \item Użytkownik edytuje wybrane pola.
        \item Użytkownik zatwierdza wprowadzone zmiany.
        \item Wprowadzone zmiany zostają zaktualizowane na liście.\\
    \end{enumerate}
    
    \item Użytkownik ma możliwość wyszukania faktury, sprzedawcy i kontrahenta \\
    \textbf{Opis: } Każdy użytkownik ma możliwość wyszukania faktury (po numerze faktury, sprzedawcy, kontrahencie), sprzedawcy oraz kontrahenta (po nazwie, NIPie i REGONie). Wprowadzona do wyszukiwarki fraza nie musi być pełna, może to być część nazwy, numeru. \\
    \textbf{Przebieg: }
    \begin{enumerate}
        \item Użytkownik wpisuje frazę w wyszukiwarce
        \item Jeżeli istnieje obiekt o podanej frazie zwracana jest lista obiektów, które pasują do podanej frazy.
        \item Jeżeli obiekt nie istnieje zostaje zwrócona pusta lista.\\
    \end{enumerate}
    
    \item Użytkonik ma możliwość wygenerowania pliku PDF z faktury \\
    \textbf{Opis: } Każdy użytkownik ma możliwość wygenerowania pliku PDF z wybranej faktury. \\
    \textbf{Przebieg pierwszy (dodanie nowej faktury): }
    \begin{enumerate}
        \item Użytkownik tworzy nową fakturę.
        \item Użytkownik musi uzupełnić wszystkie wymagane dane, w celu wygenerowania faktury (kontrahent, sprzedawca, pozycje na fakturze, data wystawienia).
        \item Użytkownik ma możliwość zmienienia szablonu faktury zaznaczając odpowiednie opcje (uwagi na fakturze, kto wystawił fakturę, osoba odpowiedzialna za odbiór faktury, faktura VAT lub pro forma).
        \item Użytkownik generuje fakturę po kliknięciu odpowiedniego przycisku.
        \item Poprawnie wygenerowana faktura otwiera się w domyślnym programie do PDFów.
    \end{enumerate}
    \textbf{Przebieg drugi (edycja istniejącej faktury): }
    \begin{enumerate}
        \item Użytkownik wybiera fakturę z listy.
        \item Użytkownik ma możliwość zmienienia szablonu faktury (w trybie podglądu faktury) zaznaczając odpowiednie opcje (uwagi na fakturze, kto wystawił fakturę, osoba odpowiedzialna za odbiór faktury, faktura VAT lub pro forma).
        \item Faktura musi mieć uzupełnione wszystkie wymagane pola, w celu wygenerowania faktury (kontrahent, sprzedawca, pozycje na fakturze, data wystawienia).
        \item Użytkownik generuje fakturę po kliknięciu odpowiedniego przycisku.
        \item Poprawnie wygenerowana faktura otwiera się w domyślnym programie do PDFów.
    \end{enumerate}
\end{enumerate}

\section{Wymagania niefunkcjonalne}
\begin{enumerate}
    \item \textbf{Wymagania estetyczne} \\
    Kolorystyka interfejsu użytkownika powinna skałdać się z zimnych barw. Interfejs ma być przyjazny dla oka użytkownika, tak aby dłuższe użytkowanie niesprawiało problemów.
    \\
    \item \textbf{Wymagania dotyczące ergonomii i wygody}
    \begin{itemize}
        \item Interfejs powinien posiadać pasek nawigacyjny, ułatwiający poruszanie się po aplikacji
        \item Aplikacja powinna być odporna na błędy - interfejs powinien zgłaszać niepowodzenie operacji oraz wychwytywać błędy i informować o nich użytkownika. \\
    \end{itemize}
    
    \item \textbf{Wymagania wydajnościowe}\\
    Każda interakcja użytkownika z interfejsem powinna być nie dłuższa niż 3 sekundy. Każdy użytkownik aplikacji oczekuje płynnego i naturalnego działania interfejsu. \\
    
    \item \textbf{Wymagania dotyczące warunków oraz środowiska pracy}\\
    Aplikacja powinna działać na maszynach z systemem operacyjnym Windows 7 i wyżej.
\end{enumerate}